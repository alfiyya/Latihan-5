\documentclass[conference]{IEEEtran}
\IEEEoverridecommandlockouts
% The preceding line is only needed to identify funding in the first footnote. If that is unneeded, please comment it out.
\usepackage{cite}
\usepackage{amsmath,amssymb,amsfonts}
\usepackage{algorithmic}
\usepackage{graphicx}
\usepackage{textcomp}
\usepackage{xcolor}
\def\BibTeX{{\rm B\kern-.05em{\sc i\kern-.025em b}\kern-.08em
    T\kern-.1667em\lower.7ex\hbox{E}\kern-.125emX}}
\begin{document}

\title{Implementasi Algoritma DIjakstra Dalam Menemukan Jarak Terdekat Dari Lokasi Pengguna Ke Tanaman Yang Di Tuju\\
{\footnotesize \textsuperscript{*}Note: Sub-titles are not captured in Xplore and
should not be used}
\thanks{Identify applicable funding agency here. If none, delete this.}
}

\author{\IEEEauthorblockN{1\textsuperscript{st} Rafli F.Amanada}
\IEEEauthorblockA{\textit{School of Electrical and Informatics} \\
\textit{Institut Teknologi Bandung}\\
Bandung,Indonesia  \\
@std.stei.itb.ac.id}


\maketitle

\begin{abstract}
Kebun Raya Purwodadi dengan luas area sekitar 85hektar ternyata kekurangan papan informasi yang menyebabkan pengunjung kerap kali kebingungan dalam mencari lokasi tana-man tertentu. Paper ini bertujuan untuk membuat simulasi
dari algoritma yang dapat menentukan jarak terdekat antara pengunjung (pengguna program) dengan lokasi tanaman yang dituju. Algoritma yang digunakan adalah algoritma Dijkstra yang beroperasi secara menyeluruh (greedy) untuk menguji
setiap persimpangan (Vertex) dan jalan (Edge) pada Kebun Raya Purwodadi. Berdasarkan hasil simulasi dan pengujian, kompleksitas ruang dari program ini adalah O(V) karena adanya
pembentukan array yang berisi V nodes untuk mencari heap min-imum. Sementara, kompleksitas waktu dari algoritma tersebut adalah O(V2).
\end{abstract}

\begin{IEEEkeywords}
Dijakstara, \Veriex, \Edge, Tanaman 
\end{IEEEkeywords}

\section{Introduction}
Studi mengenai penggunaan algoritma Dijkstra dalam mencari jarak terdekat dapat diimplementasikan pada kasus pencarian tanaman pada Kebun Raya Purwodadi seperti yang telah
dilakukan oleh Yusuf et al di tahun 2017 [1]. Paper ini bertujuan untuk melakukan simulasi kembali terhadap penelitian yang telah dilakukan dengan bahasa C serta mengevaluasi
efisiensinya melalui perhitungan kompleksitas waktu dan ruang dengan analisis Big-O. Di Kecamatan Purwodadi, Kabupaten Pasuruan, terdapat
salah satu kebun raya di Indonesia yang bernama Kebun Raya Purwodadi yang memiliki luas area hingga 85 hektar. Kebun raya sebagai fasilitas rekreasi dan penelitian ini ternyata
kekurangan papan informasi yang seharusnya disediakan oleh pihak pengelola. Hal ini menyebabkan banyaknya pengunjung
yang merasa kebingungan untuk mencari lokasi dari tanaman tertentu. Oleh karena itu, Yusuf et al (2017) memutuskan
untuk membuat suatu aplikasi dengan memanfaatkan algoritma Dijkstra untuk membantu pengunjung Kebun Raya Purwodadi dalam mencari lokasi tertentu.
Algoritma Dijkstra digunakan karena algoritma ini dapat beroperasi secara menyeluruh (algoritma greedy) terhadap
semua alternatif fungsi serta durasi eksekusi yang lebih cepat jika dibandingkan dengan algoritma serupa, yaitu Bellman-Ford. Algoritma ini akan mencari jalur dengan ’biaya’ atau
\section{Studi Pustaka}

\subsection{Algoritma Dijaktsara}
Algoritma Dijkstra adalah algoritma yang digunakan untuk 
menemukan jarak jalur terpendek antara dua vertice pada
graph berbobot dan tidak berarah sederhana [2]. Berbobot
berarti grafik memiliki edge dengan suatu ’bobot’ atau harga.
Bobot dapat merepresentasikan jarak, waktu, atau apapun
yang memodelkan koneksi antara kedua node. Tidak berarah
memiliki arti bahwa untuk setiap node yang terhubung, kita
dapat mendekati suatu node dari kedua arah. Pendekatan Di-
jikstra juga memiliki asumsi bahwa bobot pada edge memiliki
nilai yang tidak negatif. Hal ini karena nilai bobot akan
terus dibandingkan dan diambil nilai yang paling kecil. Ada
banyak varian pada algoritma ini, namun pada percobaan
ini digunakan varian dimana suatu node ditetapkan menjadi
source node. Dari node inilah akan dicari jarak terpendek
diantara node lain. Algoritma ini dicetuskan oleh Edsger
Wybe Dijkstra, salah seorang tokoh ternama di bidang com-
puter science [3]. Kompleksitas dari algoritma dijkstra adalah
O(n2), dengan n menyatakan jumlah vertice dari graph yang
bersangkutan.

\subsection{ Kebun Raya Purwodadi}
Kebun Raya Purwodadi adalah kebun penelitian di Keca-
matan Purwodadi, Jawa Timur. Ia juga dikenal dengan nama
Hortus Ilkim Kering Purwodadi dan didirikan tanggal 30 Jan-
uari 1941 oleh Dr. L.G.M. Baas Becking. Sebagai cabang dari
Kebun Raya Bogor, ia memiliki fungsi mengkoleksi tumbuhan
yang hidup di dataran rendah kering. Sebagai Balai Konservasi
Tumbuhan di bawah Pusat Konservasi Tumbuhan Kebun Raya,
Kedeputian Bidang Ilmu Pengetahuan Hayati LIPI, kebun raya
ini memiliki banyak tumbuhan yang dinaunginya. Dengan
menggunakan algoritma Dijkstra, diharapkan ia dapat mem-
bantu pengunjung mencari tanaman tertentu maupun jarak
yang paling optimal

\section{Metodologi Penelitian}
Peneliti menggunakan beberapa tahap dalam penyusunan
paper ini. Pertama, dilakukan pengkajian dan studi literatur
dengan membaca referensi paper yang berkaitan dan memilih
paper yang dapat menjadi acuan dalam penelitian yang di-
lakukan, sehingga dari pilihan topik dan tema yang berkaitan
secara luas dapat dikecilkan menjadi sebuah paper yang men-
cakup mayoritas dari topik yang dibahas. Setelah ditemukan
beberapa paper, dilakukan perangkuman untuk menentukan
paper yang sesuai sekaligus membahas poin-poin penting
dari paper yang ingin dicapai. Setelah kedua tahap tersebut
dilewati, penentuan paper yang dijadikan prototype penelitian
merupakan hal yang mudah dan menjadi titik pencapaian
dalam studi literatur dan pemilihan topik dari prototype peneli-
tian yang dilakukan.
Setelah itu, tahap selanjutnya yang dilakukan oleh peneliti
adalah pembuatan prototype berupa program yang ditulis
dalam bahasa C. Pembuatan prototype berupa kode ini di-
lakukan terus-menerus dengan menggunakan metode trial and
error sehingga perlu dilakukan revisi hingga protoype kode
yang dibuat dapat mendapatkan output yang optimal dan
sesuai dengan spesifikasi yang diharapakan. Tahap terakhir dari penelitian adalah pemaparan kode yang berhasil di-
jalankan tersebut ke dalam paper.


\section{Implementasi Dan Pengujian}
\subsection{Implementasi Graph pada Array dalam Bahasa C}
Program akan dimulai dengan pembacaan file bernama
listtanaman.txt. File tersebut akan menyimpan informasi men-
genai semua nama tanaman yang bersangkutan. Setelah pem-
bacaan tersebut, akan dicari informasi mengenai bobot graph
yang menghubungkan node. Informasi ini disimpan di dalam
matriks segitiga bawah kiri didalam file jarakantarpohon.txt
yang juga dibuka saat program dijalankan. Matriks menggam-
barkan bobot antara jarak dua node tanaman sekali saja karena
pemodelan undirected graph yang memiliki jarak sama baik
dari a ke b maupun b ke a. Nilai −1 akan menggambarkan
bagian node yang tidak terhubung sama sekali dalam graph
dan juga dinyatakan dalam suatu variabel bernama int max
(Jaraknya sebesar tak hingga). Nilai jarak terpendek akan
disimpan dalam array tersebut selagi program berjalan.



\section*{References}

\begin{thebibliography}{00}
\bibitem{b1} M. S. Yusuf, H. M. Az-Zahra, and D. H. Apriyanti, “Implementasi algoritma dijkstra dalam menemukan jarak terdekat dari lokasi pengguna ke tanaman yang di tuju berbasis android (studi kasus di kebun raya pur-wodadi),” Jurnal Pengembangan Teknologi Informasi dan Ilmu Komputer e-ISSN, vol. 2548, p. 964X, 2017
\bibitem{b2} K. H. Rosen, Discrete Mathematics and It’s Applications - Seventh Edition. McGraw-Hill.Inc, 2012.
\bibitem{b3} E. Dijkstra, A Note on Two Problems in Connection with Graphs, 1959.
\end{thebibliography}


\end{document}
